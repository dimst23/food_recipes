% Super lentils recipe
\begin{recipe}
[
    preparationtime = 15 min,
    bakingtime = 35 min,
    portion = \portion{2},
]{Lentils with Herbs}

    \introduction
    {
        Lentils need to be cooked first, so put water for boiling in a sauce pan. Lentils have to be washed well and drained before cooking to remove any debris or stones.
    }
    
    \ingredients
    {
        \unit[500]{ml} & Water \\
        \unit[200]{g} & Brown or Green Lentils\\
        1 clove & Garlic, minced \\
        0.5 large & Onion, chopped \\
        1 & Red pepper \\
        1 tbsp & Olive oil \\
        0.5 cup & Tomato sauce \\
        1.5 tbsp & Balsamic vinegar \\
        or & Apple cider vinegar \\
        1 tbsp & Shiro miso \\
        Some & Basil \\
        & Thyme \\
        & Rosemary \\
        & Black pepper \\
        A little & Curry \\
        & Parsley, chopped \\
        & Dill, chopped
    }   
    
    \preparation
    {
        \step Cook lentils first by bringing water to boil and then adding lentils. Bring back to boil and then reduce heat low and simmer, uncovered, for 20 minutes or until lentils tender.
        \step While lentils are cooking prepare the sauce. Add the tomato sauce, ready made or fresh from tomatoes, to a deep dish. Put in the basil, thyme, rosemary, black pepper and a bit of curry. Mix well and taste. Add more of the spices if you want stronger taste. Then add the Balsamic or Apple cider vinegar, stir well and taste again. If vinegar taste is too strong add some more tomato sauce. Finally add the miso and stir well until dissolved. Set the sauce aside and chop the onion, parsley, red pepper and dill and mice the garlic.
        \step Once lentils are cooked drain off any excess liquid and set aside. Heat the olive oil in a frying pan and once heated add the chopped onion, garlic and red pepper. Saute until onion is golden brown and red pepper has softened a bit.
        \step Add the lentils in the frying pan, along with the sauce and mix well, heating for 1-2 minutes. Before turning the heat off, add the chopped parsley and dill and mix well.
        \step Enjoy immediately with salads, rice or whatever you like. Store the leftovers in the refrigerator up to 4-5 days.
    }
    
    \suggestion[Serving Suggestion]
    {
        Recipe is best served with lettuce and rocket salad.
    }
\end{recipe}

\pagebreak